\documentclass[11pt]{article}




\usepackage[sfdefault]{FiraSans} %% option 'sfdefault' activates Fira Sans as the default text font
\usepackage[T1]{fontenc}
\renewcommand*\oldstylenums[1]{{\firaoldstyle #1}}

\usepackage{natbib}
\usepackage[french,english]{babel}
\usepackage{numprint}
\usepackage{multirow}
\usepackage{rotating}
\usepackage{fancyhdr}
\usepackage{booktabs}
\usepackage{multicol}
\usepackage{hyperref}\hypersetup{colorlinks=true}

\usepackage{amsmath,amssymb,amsfonts,textcomp}
\usepackage{color}
\usepackage{calc}
 \setlength{\tabcolsep}{8pt}
\usepackage{setspace}
\onehalfspacing
\usepackage{longtable}
\usepackage{graphicx}
\usepackage[margin=1in]{geometry}
\setlength{\parindent}{0pt}
\usepackage[bottom]{footmisc}
\pagestyle{fancy}
\usepackage{titlesec}
\usepackage{lipsum}
\usepackage{cancel}
\usepackage{multicol}

\usepackage{amsmath,amssymb}
\usepackage{lmodern}
\usepackage{iftex}
\ifPDFTeX
  \usepackage[T1]{fontenc}
  \usepackage[utf8]{inputenc}
  \usepackage{textcomp} % provide euro and other symbols
\else % if luatex or xetex
  \usepackage{unicode-math}
  \defaultfontfeatures{Scale=MatchLowercase}
  \defaultfontfeatures[\rmfamily]{Ligatures=TeX,Scale=1}
\fi

\titleformat{\section}
  {\normalfont\Large\scshape\bfseries}{\thesection}{1em}{}
  \titlespacing{\section}{0pt}{10pt}{0pt}
\titleformat{\subsection}
  {\normalfont\bfseries}{\thesection}{1em}{}
  \titlespacing{\subsection}{0pt}{6pt}{0pt}
\providecommand{\tightlist}{%
  \setlength{\itemsep}{0pt}\setlength{\parskip}{0pt}}\newenvironment{itemize*}%

  
\lhead{EC200 - Econometrics and Applications}

\setlength\parskip{0.10in}
\begin{document}
\thispagestyle{plain}
\singlespacing


EC200: Econometrics and Applications\hfill Fall 2021\\

%\vspace{1.5cm}
\begin{center}
\Large{\textbf{Problem Set 2}}\\
\end{center}
\bigskip


%{
%\setcounter{tocdepth}{3}
%\tableofcontents
%}

You can download the data file you need for question 4
\href{https://ec200f21.netlify.app/assignment/collegedistance.dta}{here}.

\hypertarget{what-do-i-submit}{%
\section*{What do I submit?}\label{what-do-i-submit}}

\begin{itemize}
\tightlist
\item
  Your written up answers to exercise questions. If you work on a piece
  of paper, please scan using some sort of phone software (like
  Microsoft Lens or Adobe Scan) rather than just taking a picture.
\item
  A do-file that runs your Stata analysis (for question 4).
\item
  A log file that includes the output from running your do-file (for
  question 4).
\end{itemize}

\hypertarget{exercises}{%
\section*{Exercises}\label{exercises}}

\begin{enumerate}
\def\labelenumi{\arabic{enumi}.}
\item
  The following table shows, for eight vintages of delicious wine,
  purchases per buyer (\(y\)) and the wine buyer's rating (\(x\)) in a
  given year:

  \begin{longtable}[]{@{}lcccccccc@{}}
  \toprule
  \(x\) & 3.6 & 3.3 & 2.8 & 2.6 & 2.7 & 2.9 & 2.0 & 2.6 \\
  \midrule
  \endhead
  \(y\) & 24 & 21 & 22 & 22 & 18 & 13 & 9 & 6 \\
  \bottomrule
  \end{longtable}

  \begin{enumerate}
  \item
    Estimate \emph{by hand} the regression of purchases per buyer on the
    buyer's rating.\\
  \item
    Interpret the slope of the estimated regression line.\\
  \item
    Interpret the intercept of the estimated regression line .\\
  \end{enumerate}
\end{enumerate}

\begin{enumerate}
\def\labelenumi{\arabic{enumi}.}
\setcounter{enumi}{1}
\tightlist
\item
  Suppose that a random sample of 200 20-year-old men is selected from a
  population and that these men's height and weight are recorded. A
  regression of weight (measured in pounds) on height (measured in
  inches) yields



$\widehat{Weight}=-99.41 + 3.94 Height$

$R^2 = 0.81$; $SER = 10.2$
\begin{enumerate}
\item  What is the predicted weight for someone who is 70 inches tall?
    65 inches tall?

\item One 20-year-old man has a late growth spurt and grows 1.5 inches
    over the course of the year. What is the regression's prediction
    for the increase in his weight?

\item  Suppose that you want to translate the results of this equation
    into centimeters and kilograms. What are the regression
    estimates from this new regression? Give all results, including
    estimated coefficients, $R^2$, and $SER$.

\item Interpret the $R^2$ value. Does it indicate anything about
    whether these estimates are likely to be biased? Explain.
\end{enumerate}
\end{enumerate}
\begin{enumerate}
\def\labelenumi{\arabic{enumi}.}
\setcounter{enumi}{2}
\item
  Consider the savings function:

\begin{equation*}
sav =  \beta_0 + \beta_1 inc + u, u = e\sqrt{inc}
\end{equation*}
where \(e\) is a random variable with \(E(e) = 0\) and
\(Var(e) = \sigma^2_e\). Assume that \(e\) is independent of \(inc\).

\begin{enumerate}
\def\labelenumi{\alph{enumi}.}

\item
  Show that \(E(u|inc)=0\), so that the key zero conditional mean
  assumptionis satisfied. {[}Hint: If \(e\) is independent of \(inc\),
  then \(E(e|inc) = E(e)\){]}
\item
  Show that \(Var(u|inc) = \sigma^2_einc\), so that the homoskedasticity
  assumption is violated. In particular, the variance of \(sav\)
  increases with \(inc\). {[}Hint: \(Var(e|inc) = Var(e)\) if \(inc\)
  and \(e\) are independent!{]}
\item
  Why might it be reasonable to assume that the variance of savings
  increases with family income?
\end{enumerate}
\end{enumerate}
\begin{enumerate}
\def\labelenumi{\arabic{enumi}.}
\setcounter{enumi}{3}
\item
  The data file
  \href{../collegedistance.dta}{\texttt{collegedistance.dta}} contains
  data from a random sample of high school seniors interviewed in 1980
  and re-interviewed in 1986\footnote{These data were provided by
    Professor Cecilia Rouse of Princeton University and were used in her
    paper ``Democratization or Diversion? The Effect of Community
    Colleges on Educational Attainment,'' Journal of Business and
    Economic Statistics, April 1995, 12(2): 217--224.} Use these data to
  investigate the relationship between the number of completed years of
  education for young adults and the distance from each student's high
  school to the nearest four-year college. (Proximity to college lowers
  the cost of education, so that students who live closer to a four-year
  college should, on average, complete more years of higher education.)


\begin{enumerate}
\def\labelenumi{\alph{enumi}.}
\item
  Run a regression of years of completed education (\(ED\)) on distance
  to the nearest college (\(Dist\)), where \(Dist\) is measured in tens
  of miles. (For example, \(Dist=2\) means that the distance is 20
  miles.)\footnote{Though we haven't covered regressions in our stata
    labs, I talk thorugh interpreting regression output in the
    \href{https://youtu.be/DZo5m5q3bmA}{Chapter 5 Video}. You can
    regress a dependent variable \texttt{y} on an independent variable
    \texttt{x} with the command \texttt{regress\ y\ x}}. Write the
  equation you estimated in the form
  \(\widehat{ED} = \beta_0 + \beta_1 Dist\)
\item
  How does the average value of years of completed schooling change when
  colleges are built close to where students go to high school?
\item
  Bob's high school was 20 miles from the nearest college. Predict Bob's
  years of completed education using the estimated regression. How would
  the prediction change if Bob lived 10 miles from the nearest college?
\item
  Does distance to college explain a large fraction of the variance in
  educational attainment across individuals? Explain.
\item
  Provide an example of a factor that might cause this model to violate
  the zero conditional mean assumption. Explain your reasoning.
\item
  What is the value of the standard error of the regression?\footnote{There
    are a few ways to find it in Stata's output. The easiest is to note
    that ``root MSE'' is the square root of the SER} What are the units
  for the standard error (meters, grams, years, dollars, cents, or
  something else)?
\item
  Is the estimated regression slope coefficient statistically
  significant at the 5\% level? What is the p-value associated with
  coefficient's t-statistic?
\item
  Construct a 95\% confidence interval for the slope coefficient.
\item
  Estimate a regression that restricts the sample to men, and calculate
  a 95\% confidence interval for the slope. Do the same, restricting the
  sample to women. Does it look like the effect of distance on completed
  years of education is different?\footnote{Note that we cannot make
    claims about whether they are statistically different because the
    estimates come from two different samples! A hypothesis test here
    would be awesome, but we need to build a few more skills to do this.}
\end{enumerate}
\end{enumerate}
\end{document}
