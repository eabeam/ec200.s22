\documentclass[11pt]{article}




\usepackage[sfdefault]{FiraSans} %% option 'sfdefault' activates Fira Sans as the default text font
\usepackage[T1]{fontenc}
\renewcommand*\oldstylenums[1]{{\firaoldstyle #1}}

\usepackage{natbib}
\usepackage[french,english]{babel}
\usepackage{numprint}
\usepackage{multirow}
\usepackage{rotating}
\usepackage{fancyhdr}
\usepackage{booktabs}
\usepackage{multicol}
\usepackage{hyperref}\hypersetup{colorlinks=true}

\usepackage{amsmath,amssymb,amsfonts,textcomp}
\usepackage{color}
\usepackage{calc}
 \setlength{\tabcolsep}{8pt}
\usepackage{setspace}
\onehalfspacing
\usepackage{longtable}
\usepackage{graphicx}
\usepackage[margin=1in]{geometry}
\setlength{\parindent}{0pt}
\usepackage[bottom]{footmisc}
\pagestyle{fancy}
\usepackage{titlesec}
\usepackage{lipsum}
\usepackage{cancel}
\usepackage{multicol}

\usepackage{amsmath,amssymb}
\usepackage{lmodern}
\usepackage{iftex}
\ifPDFTeX
  \usepackage[T1]{fontenc}
  \usepackage[utf8]{inputenc}
  \usepackage{textcomp} % provide euro and other symbols
\else % if luatex or xetex
  \usepackage{unicode-math}
  \defaultfontfeatures{Scale=MatchLowercase}
  \defaultfontfeatures[\rmfamily]{Ligatures=TeX,Scale=1}
\fi

\titleformat{\section}
  {\normalfont\Large\scshape\bfseries}{\thesection}{1em}{}
  \titlespacing{\section}{0pt}{10pt}{0pt}
\titleformat{\subsection}
  {\normalfont\bfseries}{\thesection}{1em}{}
  \titlespacing{\subsection}{0pt}{6pt}{0pt}

\lhead{EC200 - Econometrics and Applications}
\rhead{Version: \today}
\setlength\parskip{0.10in}
\begin{document}
\thispagestyle{plain}
\singlespacing


Version: \today \hfill Fall 2021\\
EC200: Econometrics and Applications
%\vspace{1.5cm}
\begin{center}
\Large{\textbf{Problem Set 4}}\\
\end{center}
\bigskip


\hypertarget{welcome}{%
\section*{Welcome}\label{welcome}}

Chapter 8 and 9 problems! Enjoy!

See the exercises below, or you can \href{https://ec200f21.netlify.app/assignment/04-ps.pdf}{download them as
a pdf}.

\hypertarget{what-do-i-submit}{%
\section*{What do I submit?}\label{what-do-i-submit}}

\begin{itemize}
\item
  Your written up answers to exercise questions. If you work on a piece
  of paper, please scan using some sort of phone software (like
  Microsoft Lens or Adobe Scan) rather than just taking a picture.
\item
  A do-file that runs your Stata analysis (for question 7).
\item
  A log file that includes the output from running your do-file (for
  question 7).
\end{itemize}

\hypertarget{exercises}{%
\section*{Exercises}\label{exercises}}

\begin{enumerate}
\def\labelenumi{\arabic{enumi}.}

\item
  The following equation describes the median housing price in a
  community in terms of amount of pollution (\(nox\) for nitrous oxide)
  and the average number of rooms in houses in the community
  (\(rooms\)):


\(log(price) = \beta_0 = \beta_1log(nox) + \beta_2rooms + u\)

\begin{enumerate}
\def\labelenumi{\alph{enumi}.}

\item
  What are the probable signs of \(\beta_1\) and \(\beta_2\)? What is
  the interpretation of \(\beta_1\)? Explain.
\item
  Why might \(nox\) {[}or more precisely, \(log(nox)\){]} and \(rooms\)
  be negatively correlated? If this is the case, does the simple
  regression of \(log(price)\) on \(log(nox)\) produce an upward or a
  downward biased estimator of \(\beta_1\)?
\item
  Using data, the following equations were estimated:
\end{enumerate}

\(\widehat{log(price)} = 11.71 - 1.043 log(nox)\), \(n = 506\),
\(R^2 = 0.264\)

\(\widehat{log(price)} = 9.23 - 0.718 log(nox) + 0.306 rooms\),
\(n = 506\), \(R^2 = 0.514\)

Is the relationship between the simple and multiple regression estimates
of the elasticity of \(price\) with respect to \(nox\) what you would
have predicted, given your answer in part (ii)? Does this mean that
0.718 is definitely closer to the true elasticity than 1.043?


\item   Read the box \emph{``The Return to Education and the Gender Gap''} in
  Section 8.3.


\begin{enumerate}
\def\labelenumi{\alph{enumi}.}

\item
  Consider a man with 16 years of education and 2 years of experience.
  Use the results from column (4) of Table 8.1 and the method in Key
  Concept 8.1 to estimate the expected change in the logarithm of
  average hourly earnings (AHE) associated with an additional year of
  experience.
\item
  Explain why your answer to (a) does not depend on the region he is
  from.
\item
  Repeat (a), assuming 10 years of experience.
\end{enumerate}


\item
  To answer this question, refer to \emph{Table 8.3: Nonlinear
  Regression Model of Test Scores} in your textbook:


\begin{enumerate}
\def\labelenumi{\alph{enumi}.}

\item
  A researcher suspects that the effect of \% Eligible for subsidized
  lunch has a nonlinear effect on test scores. In particular, he
  conjectures that increases in this variable from 10\% to 20\% have
  little effect on test scores but that changes from 50\% to 60\% have a
  much larger effect. i. Describe a nonlinear specification that can be
  used to model this form of nonlinearity. ii. How would you test
  whether the researcher's conjecture was better than the linear
  specification in column (7) of Table 8.3?
\item
  A researcher suspects that the effect of income on test scores is
  different in districts with small classes than in districts with large
  classes. i. Describe a nonlinear specification that can be used to
  model this form of nonlinearity.
\end{enumerate}



\item
  Labor economists studying the determinants of women's earnings
  discovered a puzzling empirical result. Using randomly selected
  employed women, they regressed earnings on the women's number of
  children and a set of control variables (age, education, occupation,
  and so forth). They found that women with more children had higher
  wages, controlling for these other factors. Explain how sample
  selection might be the cause of this result. (Hint: Notice that women
  who do not work outside the home are missing from the sample.) {[}This
  empirical puzzle motivated James Heckman's research on sample
  selection that led to his 2000 Nobel Prize in Economics. See Heckman
  (1974){]}


\item
  The demand for a commodity is given by
  \(Q = \beta_0 + \beta_1 P + u\), where \(Q\) denotes quantity, \(P\)
  denotes price, and \(u\) denotes factors other than price that
  determine demand. Supply for the commodity is given by
  \(Q = \gamma_0 + \gamma_1P + v\), where \(v\) denotes factors other
  than price that determine supply. Suppose \(u\) and \(v\) both have a
  mean of 0, have variances \(\sigma^2_u\) and \(\sigma^2_v\), and are
  mutually uncorrelated.

\begin{enumerate}
\def\labelenumi{\alph{enumi}.}

\item
  Solve the two simultaneous equations to show how Q and P depend on u
  and v.
\item
  Derive the means of P and Q.
\item
  Derive the variance of P, the variance of Q, and the covariance
  between Q and P.
\end{enumerate}


\item
  Revisit the box \emph{``The Return to Education and the Gender Gap''}
  in Section 8.3. Discuss the internal and external validity of the
  estimated effect of education on earnings.
\item
  Complete
  \href{https://www.princeton.edu/~mwatson/Stock-Watson_3u/Students/AEE/Stock_Watson_3U_AEE_8_2.pdf}{Additional
  Empirical Exercise 8.2} using the dataset
  \href{https://www.princeton.edu/~mwatson/Stock-Watson_3u/Students/EE_Datasets/CollegeDistance.dta}{\texttt{CollegeDistance.dta}}
\end{enumerate}

\end{document}
