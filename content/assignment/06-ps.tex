2% Options for packages loaded elsewhere
\PassOptionsToPackage{unicode}{hyperref}
\PassOptionsToPackage{hyphens}{url}
%
\documentclass[
]{article}
\usepackage{lmodern}
\usepackage{amssymb,amsmath}
\usepackage{ifxetex,ifluatex}
\ifnum 0\ifxetex 1\fi\ifluatex 1\fi=0 % if pdftex
  \usepackage[T1]{fontenc}
  \usepackage[utf8]{inputenc}
  \usepackage{textcomp} % provide euro and other symbols
\else % if luatex or xetex
  \usepackage{unicode-math}
  \defaultfontfeatures{Scale=MatchLowercase}
  \defaultfontfeatures[\rmfamily]{Ligatures=TeX,Scale=1}
\fi
% Use upquote if available, for straight quotes in verbatim environments
\IfFileExists{upquote.sty}{\usepackage{upquote}}{}
\IfFileExists{microtype.sty}{% use microtype if available
  \usepackage[]{microtype}
  \UseMicrotypeSet[protrusion]{basicmath} % disable protrusion for tt fonts
}{}
\makeatletter
\@ifundefined{KOMAClassName}{% if non-KOMA class
  \IfFileExists{parskip.sty}{%
    \usepackage{parskip}
  }{% else
    \setlength{\parindent}{0pt}
    \setlength{\parskip}{6pt plus 2pt minus 1pt}}
}{% if KOMA class
  \KOMAoptions{parskip=half}}
\makeatother
\usepackage{xcolor}
\IfFileExists{xurl.sty}{\usepackage{xurl}}{} % add URL line breaks if available
\IfFileExists{bookmark.sty}{\usepackage{bookmark}}{\usepackage{hyperref}}
\hypersetup{
  pdftitle={Problem set 6},
  hidelinks,
  pdfcreator={LaTeX via pandoc}}
\urlstyle{same} % disable monospaced font for URLs
\setlength{\emergencystretch}{3em} % prevent overfull lines
\providecommand{\tightlist}{%
  \setlength{\itemsep}{0pt}\setlength{\parskip}{0pt}}
\setcounter{secnumdepth}{-\maxdimen} % remove section numbering
\ifluatex
  \usepackage{selnolig}  % disable illegal ligatures
\fi

\title{Problem set 6}
\author{}
\date{Due 2020-11-20}

\begin{document}
\maketitle

\hypertarget{welcome}{%
\subsection{Welcome}\label{welcome}}

Our final problem set, covering chapters 10 and 12

See the exercises below

\hypertarget{what-do-i-submit}{%
\subsection{What do I submit?}\label{what-do-i-submit}}

\begin{itemize}
\tightlist
\item
  Your written up answers to exercise questions. If you work on a piece
  of paper, please scan using some sort of phone software (like
  Microsoft Lens or Adobe Scan) rather than just taking a picture.
\item
  A do-file that runs your Stata analysis.
\item
  A log file that includes the output from running your do-file.
\end{itemize}

\hypertarget{exercises}{%
\subsection{Exercises}\label{exercises}}

\begin{enumerate}
\def\labelenumi{\arabic{enumi}.}
\item
  In 1985, neither Florida nor Georgia had laws banning open alcohol
  containers in vehicle passenger compartments. By 1990, Florida had
  passed such a law, but Georgia had not.

  \begin{enumerate}
  \def\labelenumii{\alph{enumii}.}
  \item
    Suppose you collect random samples of the driving-age population in
    both states, for 1985 and 1990. Let \(arrest\) be a binary variable
    equal to one if a person was arrested for drunk driving during the
    year. Without controlling for any other factors, write down a linear
    probability model that allows you to test whether the open container
    law reduced the probability of being arrested for drunk driving.
    Which coefficient measures the effect of the law?
  \item
    Why might you want to control for other factors in the model? What
    might some of these factors be?
  \item
    Now, suppose that you can only collect data for 1985 and for 1990 at
    the county level for the two states. The dependent variable would be
    the fraction of licensed drivers arrested for drunk driving during
    the year. How does this data structure differ from the
    individual-level data described in part (a)? What econometric method
    would you use?
  \end{enumerate}
\item
  For this exercise, use
  \href{../materials/JTRAIN.dta}{\texttt{JTRAIN.dta}} to determine the
  effect of a job training grant on hours of job training per employee.
  The basic model for the three years is the following: \[\begin{split} 
  hrsemp_{it} &= \beta_0 + \delta_1 d88_t + \delta_2 d89_t +\\
  &  \beta_1 grant_{it} + \beta_2 grant_{i,t-1} + \beta_3 log(employ_{it}) + a_i + u_{it}
  \end{split}\]

  \begin{enumerate}
  \def\labelenumii{\alph{enumii}.}
  \item
    Estimate the equation using first differencing. How many firms are
    used in the estimation? How many total observations would be used if
    each firm had data on all variables (in particular, \(hrsemp\)) for
    all three time periods?
  \item
    Interpret the coefficient on \(grant\), and comment on its
    significance.
  \item
    Is it surprising that \(grant_{-1}\) is insignificant? Explain.
  \item
    Do larger firms train their employees more or less, on average? How
    big are the differences in training?
  \end{enumerate}
\item
  Use \href{../materials/CRIME4.dta}{\texttt{CRIME4.dta}} for this
  exercise, and see example 13.9 in this poor-quality scanned
  \href{../materials/example-13.9.pdf}{upload}.

  \begin{enumerate}
  \def\labelenumii{\alph{enumii}.}
  \item
    Replicate the results in Example 13.9.
  \item
    Re-estimate the unobserved effects model for crime in Example 13.9,
    but use fixed effects rather than differencing. Are there any
    notable sign or magnitude changes in the coefficients? What about
    statistical significance?
  \item
    Add the logs of each wage variable in the data set and estimate the
    model by fixed effects. How does including these variables affect
    the coefficient on the criminal justic variables in part (b)?
  \item
    Do the wage variables in part (c) have the expected sign? Are they
    jointly significant?
  \end{enumerate}
\item
  \textbf{SW-12.6} In an instrumental variable regression model with one
  reressor, \(X_i\), and one instrument, \(Z_i\), the regression of
  \(X_i\) onto \(Z_i\) has \(R^2 = 0.05\) and \(n = 100\). Is \(Z_i\) a
  strong instrument?\footnote{\emph{Hint:} See equation 7.14 in your
    textbook.} Would your answer change if \(R^2 = 0.05\) and
  \(n = 500\)?
\end{enumerate}

\begin{enumerate}
\def\labelenumi{\arabic{enumi}.}
\setcounter{enumi}{4}
\tightlist
\item
  \textbf{SW-12.9} A researcher is interested in the effect of military
  service on human capital. She collects data from a random sample of
  4000 workers aged 40 and runs the OLS regression
  \(Y_i = \beta_0 + \beta_1X_i + u_i\), where \(Y_i\) is a worker's
  annual earnings and \(X_i\) is a binary variable equal to 1 if the
  person served in the military and is equal to 0 otherwise.

  \begin{enumerate}
  \def\labelenumii{\alph{enumii}.}
  \tightlist
  \item
    Explain why the OLS estimates are likely to be unreliable.
    (\emph{Hint:} Which variables are omitted from the reression? Are
    they correlated with military service?)
  \item
    During the Vietnam war there was a draft in which priority for the
    draft was determined by a national lottery. The days of the year
    were randomly re-ordered 1 through 365. (Those whose birthdays were
    ordered first were drafted before those with birthdates ordered
    second, and so forth.) Explain how the lottery might be used as an
    instrument to estimate the effect of military service on earnings.
    For more about this issue, see Joshua D Angrish's paper ``Lifetime
    Earnings and the Vietnam Era Draft Lottery: Evidence from Social
    Security Administration Records,'' \emph{American Economic Review},
    June 1990: 313--336.
  \end{enumerate}
\end{enumerate}

\begin{enumerate}
\def\labelenumi{\arabic{enumi}.}
\setcounter{enumi}{5}
\item
  \textbf{SW-E12.2} Does viewing a violent movie lead to violent
  behavior? If so the incidence of violent crimes, such as assault,
  should rise following the release of a violent movie that attracts
  many viewers. Alternatively, movie viewing may substitute for other
  activities, such as alcohol consumption, that lead to violent
  behavior, so that assaults should fall more when more viewers are
  attracted to the cinema. Use the data file
  \href{../materials/Movies.dta}{\texttt{Movies.dta}}, which contains data on the
  number of assaults and movie attendance for 516 weekends from 1995
  through 2004\footnote{These are aggregated versions of data provided
    by Gordon Dahl and Stefano DellaVigna, used in their paper,
    \href{https://eml.berkeley.edu//~sdellavi/wp/moviescrime08-08-01Forthc.pdf}{``Does
    Movie Volence Increase Violent Crime?''}}. A detailed description is
  given \href{../materials/movies_description.pdf}{here}. The data set
  includes weekend US attendance for strongly violent movies (such as
  \emph{Hannibal}), mildly violent movies (such as \emph{Spiderman}),
  and non-violent movies (such as \emph{Finding Nemo}). The data also
  includes the count of the number of assaults for the same weekend in a
  subset of counties in the United States. Finally, the data set
  includes indicators for year, month, whether the weekend is a holiday,
  and various measures of the weather.

  \begin{enumerate}
  \def\labelenumii{\alph{enumii}.}
  \item
    Regress the logarithm of the number of assaults
    (\(ln\_assaults= ln(assaults)\)) on the year and month indicators.
    Is there evidence of seasonality in assaults? That is, do there tend
    to be more assaults in some months than others? Explain.
  \item
    Now, regress total movie attendance
    (\(attend = attend\_v + attend\_m + attend\_n\)) on the year and
    month indicators. Is there evidence of seasonality in movie
    attendance? Explain.
  \item
    Regress \(ln\_assaults\) on \(attend\_v\), \(attend\_m\),
    \(attend\_n\), the year and month indicators, and the weather and
    holiday control variables available in the data set.

    \begin{enumerate}
    \def\labelenumiii{\arabic{enumiii}.}
    \tightlist
    \item
      Based on the regression, does viewing a strongly violent movie
      increase or decrease assaults? By how much? Is the estimated
      affect statistically significant?
    \item
      Does attendance at strongly violent movies affect us all
      differently than attendance at moderately violent movies?
      Differently than attendance at non-violent movies?
    \item
      A strongly violent blockbuster movie is released and the weekends
      attendance is at strongly violent movies increases by 6 million;
      meanwhile, attendance falls by 2 million for moderately violent
      movies and by 1 million for non-violent movies. What is the
      predicted effect on assault? Construct a 95\% confidence interval
      for the change in assault.\footnote{\emph{Hint:} Review section
        7.3 and material surrounding equations 8.7 and 8.8}
    \end{enumerate}
  \item
    It is difficult to control for all the variables that affect
    assaults and that might be correlated with movie attendance. For
    example, the effect of the weather on assaults and movie attendance
    is only crudely approximated by the weather variables in the data
    set. However, the data set does include a set of instruments
    \(pr\_attend\_v\), \(pr\_attend\_m\), and \(pr\_attend\_n\), that
    are correlated with attendance but are (arguably) uncorrelated with
    weekend-specific factors such as the weather that affects both
    assaults add movie attendance. These instruments use historical
    attendance patterns, not information on a particular weekend, to
    predict a film's attendance in a given weekend. For example, if a
    film's attendance is high in the second week of its release, then
    this could be used to predict that attendance was also high in the
    first week of its release. The details of the construction of these
    instruments are available on in the
    \href{https://eml.berkeley.edu//~sdellavi/wp/moviescrime08-08-01Forthc.pdf}{Dahl
    and DellaVina paper}. Run the regression from part c, including
    year, month, holiday, and weather controls, but now using the
    instruments for attendance. Use this regression to now re-answer the
    questions from part c: c(1)- c(3).
  \item
    The intuition underlined the instruments in part 4 is that
    attendance in a given week is correlated with attendance and
    surrounding weeks. For each movie category, the data set includes
    attendance in surrounding weeks. Run the regression using the
    instruments \(attend\_v\_f\), \(attend\_m\_f\), \(attend\_n\_f\),
    \(attend\_v\_b\), \(attend\_m\_b\), and \(attend\_n\_b\) instead of
    the instruments used in part d, then use this regression to answer
    part c: c(1)- c(3).
  \item
    Based on your analysis, what do you conclude about the effects of
    violent movies on short-run violent behavior?
  \end{enumerate}
\end{enumerate}

\end{document}
