\documentclass[11pt]{article}




\usepackage[sfdefault]{FiraSans} %% option 'sfdefault' activates Fira Sans as the default text font
\usepackage[T1]{fontenc}
\renewcommand*\oldstylenums[1]{{\firaoldstyle #1}}

\usepackage{natbib}
\usepackage[french,english]{babel}
\usepackage{numprint}
\usepackage{multirow}
\usepackage{rotating}
\usepackage{fancyhdr}
\usepackage{booktabs}
\usepackage{multicol}
\usepackage{hyperref}\hypersetup{colorlinks=true}

\usepackage{amsmath,amssymb,amsfonts,textcomp}
\usepackage{color}
\usepackage{calc}
 \setlength{\tabcolsep}{8pt}
\usepackage{setspace}
\onehalfspacing
\usepackage{longtable}
\usepackage{graphicx}
\usepackage[margin=1in]{geometry}
\setlength{\parindent}{0pt}
\usepackage[bottom]{footmisc}
\pagestyle{fancy}
\usepackage{titlesec}
\usepackage{lipsum}
\usepackage{cancel}
\usepackage{multicol}

\usepackage{amsmath,amssymb}
\usepackage{lmodern}
\usepackage{iftex}
\ifPDFTeX
  \usepackage[T1]{fontenc}
  \usepackage[utf8]{inputenc}
  \usepackage{textcomp} % provide euro and other symbols
\else % if luatex or xetex
  \usepackage{unicode-math}
  \defaultfontfeatures{Scale=MatchLowercase}
  \defaultfontfeatures[\rmfamily]{Ligatures=TeX,Scale=1}
\fi

\titleformat{\section}
  {\normalfont\Large\scshape\bfseries}{\thesection}{1em}{}
  \titlespacing{\section}{0pt}{10pt}{0pt}
\titleformat{\subsection}
  {\normalfont\bfseries}{\thesection}{1em}{}
  \titlespacing{\subsection}{0pt}{6pt}{0pt}
\providecommand{\tightlist}{%
  \setlength{\itemsep}{0pt}\setlength{\parskip}{0pt}}\newenvironment{itemize*}%

  
\lhead{EC200 - Econometrics and Applications}
\rhead{Version: \today}
\setlength\parskip{0.10in}
\begin{document}
\thispagestyle{plain}
\singlespacing


Version: \today \hfill Fall 2021\\
EC200: Econometrics and Applications
%\vspace{1.5cm}
\begin{center}
\Large{\textbf{Lab 7: Difference in differences and panel data}}\\
\end{center}
\bigskip



\hypertarget{objectives}{%
\section*{Objectives}\label{objectives}}
\addcontentsline{toc}{subsection}{Objectives}

There are two separate parts to this lab - a set of data for working
with difference-in-differences models, and another set for working with
fixed-effects models

By the end of this lab, you should be able to complete the following
tasks in Stata:

\begin{itemize}
\item
  Estimate and interpret difference-in-differences models
\item
  Estimate panel data models using dummy variables
\item
  Interpret panel data models
\end{itemize}

\hypertarget{key-commands}{%
\section*{Key commands}\label{key-commands}}
\addcontentsline{toc}{subsection}{Key commands}

\hypertarget{using-xtset-and-xtreg}{%
\subsection*{\texorpdfstring{Using \texttt{xtset} and
\texttt{xtreg}:}{Using xtset and xtreg:}}\label{using-xtset-and-xtreg}}

The \texttt{xtset} command will tell Stata that you have panel data. For
example, if you have state and year data, then you would enter
\texttt{xtset\ state\ year}, or whatever the appropriate variable names
are.

General format: \texttt{xtset\ panelvar\ timevar}

After declaring your panel with \texttt{xtset}:

\begin{itemize}
\tightlist
\item
  Ask Stata to do a panel-regression by using xtreg instead of regress.
  Everything else proceeds as normal.
\item
  You have to tell Stata that you want to estimate a fixed effects model,
  so you add ,fe as an ``option''
\end{itemize}

For example, something like this: \texttt{xtreg\ income\ education,fe}
would regress income on education, and include fixed effects, where the
fixed effects are the \texttt{panelvar} variable you declared.

\hypertarget{adding-other-fixed-effects}{%
\subsection*{Adding other fixed
effects:}\label{adding-other-fixed-effects}}

You can add fixed effects to a model more generally with the \texttt{i.}
prefix or \texttt{areg}. A few examples:

\begin{verbatim}
xi: reg income i.educ i.bpl, robust     
reg income i.educ i.bpl, robust 

areg income i.educ,robust abosrb(bpl)
\end{verbatim}

\begin{enumerate}
\def\labelenumi{\arabic{enumi}.}
\tightlist
\item
  \texttt{xi:} this prefix is necessary for adding \texttt{i.} variables
  if the variables are in string form. Of course, you can use it any
  time! You can also use it to do fancier interactions with fixed
  effects, like \texttt{xi:\ reg\ income\ i.educ*i.bpl,\ robust}
\item
  You can exclude the prefix and just do \texttt{i.var} to create a
  bunch of indicator variables so long as your variable is
  \emph{numeric}
\item
  You can use \texttt{areg} to ``absorb'' a set of fixed effects - they
  will not be reported in your output, but they will be estimated. This
  method is less efficient than \texttt{xtreg} becuase you're using up
  degrees of freedom.
\end{enumerate}

\hypertarget{exercises}{%
\section*{Exercises}\label{exercises}}

\hypertarget{part-a-differences-in-differences}{%
\subsection*{Part A:
Differences-in-differences}\label{part-a-differences-in-differences}}

This part of the lab looks at a simple difference-in-differences model
based on Richardson and Troost (2009).\footnote{Based on Chapter 5 of
  \emph{Mastering 'Metrics}}

Mississippi is split between two Federal Reserve Districts. During the
early years of the Great Depression, the each district took a different
approach to bank runs. The Sixth District increased lending,while the
Eighth District responded by restricting lending to threatened banks. We
look at the impact of these policies on bank survival rates using
difference-in-differences.

\begin{enumerate}
\def\labelenumi{\arabic{enumi}.}
\item
  Start a new do-file and change directory to your working directory .
\item
  In your do-file, start a log and open
  \href{../materials/banks.dta}{\texttt{banks.dta}}
\item
  Using pencil \& paper or electronic means of your choosin ( ie, you
  don't need to do this in Stata), plot a graph of the number of banks
  in business, by district, by year. Plot number of banks in business on
  the y axis and the year on the x axis. Include only the years 1930 \&
  1931. Draw separate lines for the numbers of banks in District 6 and
  District 8. Draw a dotted ``counterfactual'' line based on your
  understanding of the change in bank policies. Mark all (4) actual
  values clearly
\item
  First, we're going to calculate a difference-in-difference estimator
  by hand between 1930 and 1931. Using the \texttt{browse} command, fill
  in \(x\) values from the following table:\\
\end{enumerate}

\begin{longtable}[]{@{}llll@{}}
\toprule
Number of banks in business & & & \\
\midrule
\endhead
District & 1930 & 1931 & 1931-1930 \\
District 6 & x & x & x \\
District 8 & x & x & x \\
& & & \\
District 8 - District 6 & x & x & x \\
\bottomrule
\end{longtable}

What is the difference-in-difference estimator?

\begin{enumerate}
\def\labelenumi{\arabic{enumi}.}
\setcounter{enumi}{4}
\item
  Now, generate a variable \texttt{treat}, a binary variable equal to 1
  for District 8 and 0 otherwise, and a variable \texttt{post}, a binary
  variable equal to 1 for the year 1931 or greater. Generate
  \texttt{treatXpost\ =\ treat*post}.
\item
  Using the above variables, estimate the impact of looser lending
  restrictions on the number of banks using a difference-in-difference
  estimator, restricting the sample to 1930 and 1931. Write your
  estimates, in equation form.
\item
  Now estimate the same regression, but use all years. What is the
  overall impact of looser lending restrictions on bank survival? Write
  your estimates, in equation form.
\item
  State clearly the assumption needed to interpret these
  difference-in-difference estimators as causal.
\end{enumerate}

\hypertarget{part-b---fixed-effects}{%
\subsection*{Part B - Fixed effects}\label{part-b---fixed-effects}}

Next, we're going to look at the relationship between marijuana use and
employment based on the National Longitudinal Survey of Youth 1997
Cohort (NSLY97).

\begin{enumerate}
\def\labelenumi{\arabic{enumi}.}
\item
  Download the NSLY97 data files and save them to your working
  directory.
\item
  Start a new do-file and change directory to your working directory .
\item
  In your do-file, start a log and open \texttt{nsly\_marijuana.dta}
\item
  How many individuals are in the data? How many years are they in the
  data?
\item
  Estimate a regression of whether marijuana use affects income, with no
  additional controls. Report your results in equation form.
\item
  Estimate a regression of whether marijuana use affects income, but add
  any controls you deem important (from the relatively limited selection
  I provide). How do the results change? Report your results in equation
  form.
\item
  One way to estimate fixed effects models is to use \texttt{xtreg} with
  the \texttt{,fe} option. Use \texttt{xtset} to tell Stata you have
  panel data. Then, estimate a fixed-effects regression of whether
  marijuana use affects income, with no additional controls. Include
  both individual-level and year-level fixed effects. Cluster your
  standard errors at the individual (\(id\)) level.
\item
  What is the coefficient on marijuana usage? What is the
  interpretation?
\item
  After adding fixed effects, should you include controls for gender and
  race/ethnicity to reduce omitted variable bias? Why or why not?
\item
  How do your results in part 9 using fixed effects compare to your
  results in parts 5 and 6? Why do they differ?
\item
  Name one specific factor that would create omitted variable bias in
  this regression.
\end{enumerate}

\end{document}
