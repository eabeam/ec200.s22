\documentclass[11pt]{article}




\usepackage[sfdefault]{FiraSans} %% option 'sfdefault' activates Fira Sans as the default text font
\usepackage[T1]{fontenc}
\renewcommand*\oldstylenums[1]{{\firaoldstyle #1}}

\usepackage{natbib}
\usepackage[french,english]{babel}
\usepackage{numprint}
\usepackage{multirow}
\usepackage{rotating}
\usepackage{fancyhdr}
\usepackage{booktabs}
\usepackage{multicol}
\usepackage{hyperref}\hypersetup{colorlinks=true}

\usepackage{amsmath,amssymb,amsfonts,textcomp}
\usepackage{color}
\usepackage{calc}
 \setlength{\tabcolsep}{8pt}
\usepackage{setspace}
\onehalfspacing
\usepackage{longtable}
\usepackage{graphicx}
\usepackage[margin=1in]{geometry}
\setlength{\parindent}{0pt}
\usepackage[bottom]{footmisc}
\pagestyle{fancy}
\usepackage{titlesec}
\usepackage{lipsum}
\usepackage{cancel}
\usepackage{multicol}

\usepackage{amsmath,amssymb}
\usepackage{lmodern}
\usepackage{iftex}
\ifPDFTeX
  \usepackage[T1]{fontenc}
  \usepackage[utf8]{inputenc}
  \usepackage{textcomp} % provide euro and other symbols
\else % if luatex or xetex
  \usepackage{unicode-math}
  \defaultfontfeatures{Scale=MatchLowercase}
  \defaultfontfeatures[\rmfamily]{Ligatures=TeX,Scale=1}
\fi

\titleformat{\section}
  {\normalfont\Large\scshape\bfseries}{\thesection}{1em}{}
  \titlespacing{\section}{0pt}{10pt}{0pt}
\titleformat{\subsection}
  {\normalfont\bfseries}{\thesection}{1em}{}
  \titlespacing{\subsection}{0pt}{6pt}{0pt}
\providecommand{\tightlist}{%
  \setlength{\itemsep}{0pt}\setlength{\parskip}{0pt}}\newenvironment{itemize*}%

  
\lhead{EC200 - Econometrics and Applications}
\rhead{Version: \today}
\setlength\parskip{0.10in}
\begin{document}
\thispagestyle{plain}
\singlespacing


Version: \today \hfill Fall 2021\\
EC200: Econometrics and Applications
%\vspace{1.5cm}
\begin{center}
\Large{\textbf{Lab 8: Instrumental variables}}\\
\end{center}
\bigskip



It's our final lab of the semester! 

\hypertarget{materials}{%
\section*{Materials}\label{materials}}
\addcontentsline{toc}{subsection}{Materials}

\begin{itemize}
\item
  \href{https://ec200f21.netlify.app/assignment/materials/voucher.dta}{\texttt{voucher.dta}}
\item
  Do-file template
  \href{https://ec200f21.netlify.app/assignment/materials/labtemplate_f21.do}{\texttt{labtemplate.do}}
\end{itemize}

\hypertarget{objectives}{%
\section*{Objectives}\label{objectives}}
\addcontentsline{toc}{subsection}{Objectives}

By the end of this lab, you should be able to complete the following
tasks in Stata:

\begin{itemize}
\item
  Estimate instrumental variable specifications and interpret them.
\item
  Output regression results using \texttt{outreg2}
\end{itemize}

\hypertarget{key-commands}{%
\section*{Key commands}\label{key-commands}}
\addcontentsline{toc}{subsection}{Key commands}

\hypertarget{conducting-instrumental-variables-regressions-with-ivregress}{%
\subsection*{\texorpdfstring{Conducting instrumental variables
regressions with
\texttt{ivregress}}{Conducting instrumental variables regressions with ivregress}}\label{conducting-instrumental-variables-regressions-with-ivregress}}

We can estimate an instrumental variables regression with
\texttt{ivregress}

General form:

\begin{verbatim}
ivregress estimator depvar [varlist1] (varlist2 = varlist_iv) [if] [in] 
      						 [weight] [, options]
\end{verbatim}

\begin{itemize}
\tightlist
\item
  \texttt{estimator} is where we will type \texttt{2sls}
\item
  \texttt{depvar} is your dependent variable
\item
  You can include other explanatory variables before or after the
  parentheses, `{[}varlist1{]}
\item
  In the parentheses, write you endogenous (\(x\)) then your instrument
  (\(z\)) - these can be lists!
\item
  The rest of it is just as you're used to
\end{itemize}

Example:

To estimate the following two-stage least squares equation:
\[ rent = \beta_0 + \beta_1 \widehat{hsngval} + \beta_2 pcturban + u\]
where \(\widehat{hsngval}\) is predicted from the following first-stage
equation
\[ hsngval = \alpha_0 + \alpha_1 faminc + \alpha_2 pcturban + v \]

\begin{verbatim}
webuse hsng2

ivregress 2sls rent  (hsngval = faminc ) pcturban, robust
\end{verbatim}

You can add \texttt{,\ first} to report the first-stage results:

\begin{verbatim}
`ivregress 2sls rent  (hsngval = faminc ) pcturban, robust first`
\end{verbatim}

\hypertarget{outputting-your-results-with-outreg2}{%
\subsection*{\texorpdfstring{Outputting your results with
\texttt{outreg2}}{Outputting your results with outreg2}}\label{outputting-your-results-with-outreg2}}

We are very good at reading raw Stata output. But, raw stata output has
no place in our papers. How do we make it pretty? There are lots of
ways, including \texttt{putexcel}, which lets you create customizable
excel tables with your outputs (good for descriptive statistics), and
\texttt{estout}, which does the same thing but is more regression
oriented.

Personally, I like \texttt{outreg2}, because it's easy to set up and
use. So that's what we'll use!

\texttt{outreg2} is a user-created package, which means you have to
install it:

\begin{verbatim}
ssc install outreg2
\end{verbatim}

You only have to do this once.

You'll run \texttt{outreg2} after estimating a regression. It takes your
results and saves them to a table. You can run it multiple time and
generate columns of results within the same excel sheet, which is pretty
handy! The general format of outreg2 is this:

\begin{verbatim}
// You can copy and paste this into stata, and it should work! 
//But note that it will save to your working directory

sysuse auto,clear

// Specification 1
regress mpg foreign weight headroom trunk length turn displacement
outreg2 using myfile.xls, replace 

// Specification 2  (add on)
regress mpg foreign weight headroom trunk length turn displacement,robust
outreg2 using myfile.xls, append 
\end{verbatim}

You can customize, with lots of options! (see \texttt{help\ outreg2}, or
check out \href{https://thedatahall.com/stata-outreg2-part1/}{these
resources})

What sort of things?

\begin{itemize}
\tightlist
\item
  Export directly to Word

  \begin{itemize}
  \tightlist
  \item
    \texttt{outreg2\ using\ myfile,\ word\ replace}
  \end{itemize}
\item
  Add summary statistics and p-values

  \begin{itemize}
  \tightlist
  \item
    See
    \href{https://thedatahall.com/reporting-publication-style-regression-output-in-stata-part-2/}{here}
    for more details
  \end{itemize}
\item
  Add notes

  \begin{itemize}
  \tightlist
  \item
    \texttt{outreg2\ using\ myfile,\ addnote(Dummy\ variables\ not\ shown)}
  \end{itemize}
\item
  Report only some variables

  \begin{itemize}
  \tightlist
  \item
    \texttt{outreg2\ using\ myfile,\ keep(mpg\ foreign)}
  \end{itemize}
\item
  Modify number of decimal places

  \begin{itemize}
  \tightlist
  \item
    \texttt{outreg2\ using\ myfile,\ dec(5)}
  \end{itemize}
\item
  You can use a loop to make a whole set of columns!
\end{itemize}

An example:

\begin{verbatim}
 sysuse auto,clear
 local r "replace"
   forval num=1/5 {
       regress mpg weight headroom if rep78==`num'
       sum mpg if rep78 == `num'
       local mean = `r(mean)'
       outreg2 using myfile.xls, `r' keep(headroom) title("Sample Graph") 
       			nocons addtext("Rep78", `num') addstat("Mean", `mean') auto(2) bracket
   
   local r "append"
   }
\end{verbatim}

\hypertarget{exercises}{%
\section*{Exercises}\label{exercises}}

Today we're going to work with
\href{../materials/voucher.dta}{\texttt{voucher.dta}}, a data set of
student performance from Rouse (1998). She measures the impact of
private school vouchers on student achievement. The final measure of
student performance we're interested in is \texttt{mnce}, their math
test scores in 1994 (after up to four years in the private school). We
also have some measures of baseline performance, their math test score
in 1990 (\texttt{mnce90}). The variable \texttt{choiceyrs} is the number
of years enrolled in a private school, and \texttt{selectyrs} is the
number of years a student was \emph{selected} to receive a voucher to
fund enrolling in a private school.\\

\begin{enumerate}
\def\labelenumi{\arabic{enumi}.}
\item
  In your do-file, start a log and open \texttt{voucher.dta}.
\item
  Summarize your data. Of the 990 students in the sample, how many were
  never awarded a voucher? How many had a voucher for all four years?
  How many actually attended a choice school for four years?
\item
  Predict the relationship between choice school attendance and math
  scores by regressing math scores \texttt{mnce} (dependent variable) on
  number of years enrolled in a choice school \texttt{choiceyrs}
  (independent variable). What do you find? Is this what you expect?
  What happens if you add in the variables \texttt{black},
  \texttt{hispanic}, and \texttt{female}? Write your results in equation
  form.
\item
  Why might \texttt{choiceyrs} be endogenous? Explain:
\item
  Now, estimate a regression of \(choiceyrs\) (dependent variable) on
  \(selectyrs\) (independent variable), including race/ethnicity and
  gender controls. Why is this a reasonable choice of an instrument?
  What is the F-statistic on \texttt{selectyrs}? (\emph{Hint: You can
  use the \texttt{testparm} command for a hypothesis test with just one
  coefficient})
\item
  Based on the previous regression, use the \texttt{predict} command to
  generate a predicted \(\widehat{choiceyrs}\). Estimate the regression
  of \(mnce\) on \(\widehat{choiceyrs}\), including race/ethnicity and
  gender controls. Write the estimated equation. How does your result
  compare to your OLS estimate?)
\item
  Re-estimate a regression of \(mnce\) (dependent variable) on
  \(choiceyrs\) (independent variable) using \(selectyrs\) as an
  instrument for \(choiceyrs\). However, this time, estimate the
  equation in one command line using \texttt{ivregress\ 2sls}. How do
  your results change, if at all?
\item
  Repeat your IV analysis, but this time include a control for baseline
  achievement by adding \(mnce90\). Write the results in equation form
  below. Do you find these results convincing? Explain.
\item
  We can also use multiple instruments for multiple endogenous
  variables. The variables \(choiceyrs1\), \(choiceyrs2\), etc. are
  dummy variables indicating the different number of years a student
  could have been in a choice school. Similarly, \(selectyrs1\),
  \(selectyrs2\), etc. have a similar definition, but for being selected
  from the lottery.

  Estimate the following equation using IV. \[\begin{split}
   mnce &= \beta_0 + \beta_1 choiceyrs_1 + \beta_2 choiceyrs_2 + \beta_3 choiceyrs_3 + \beta_4 choiceyrs_4 + \\
   &  \beta_5 black + \beta_6 hispanic + \beta_7 female + \beta_8 mnce90 + u
   \end{split}\]
\item
  Finally, go back through your regressions in your do-file. After each
  regression (there should be six: OLS without controls, OLS with
  controls, IV by hand, IV using \texttt{ivregress}, IV with \(mnce90\),
  and IV with multiple instruments), add a line of code to output the
  results to a word or excel file using \texttt{outreg2}.

  \textbf{Include a table with your results with your submission} -
  there should be six columns in one table. \emph{Note that you can use
  the \texttt{append} option to add each regression as a new column,
  rather than a new file.}
\end{enumerate}

References: Rouse, Cecilia Elena (1998), ``Private School Vouchers and
Student Achievement: An Evaluation of the Milwaukee Parental Choice
Program,'' \emph{The Quarterly Journal of Economics} 113(2), 553-602.

\end{document}
