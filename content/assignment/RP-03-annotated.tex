\documentclass[11pt]{article}




\usepackage[sfdefault]{FiraSans} %% option 'sfdefault' activates Fira Sans as the default text font
\usepackage[T1]{fontenc}
\renewcommand*\oldstylenums[1]{{\firaoldstyle #1}}

\usepackage{natbib}
\usepackage[french,english]{babel}
\usepackage{numprint}
\usepackage{multirow}
\usepackage{rotating}
\usepackage{fancyhdr}
\usepackage{booktabs}
\usepackage{multicol}
\usepackage{hyperref}\hypersetup{colorlinks=true}

\usepackage{amsmath,amssymb,amsfonts,textcomp}
\usepackage{color}
\usepackage{calc}
 \setlength{\tabcolsep}{8pt}
\usepackage{setspace}
\onehalfspacing
\usepackage{longtable}
\usepackage{graphicx}
\usepackage[margin=1in]{geometry}
\setlength{\parindent}{0pt}
\usepackage[bottom]{footmisc}
\pagestyle{fancy}
\usepackage{titlesec}
\usepackage{lipsum}
\usepackage{cancel}
\usepackage{multicol}

\usepackage{amsmath,amssymb}
\usepackage{lmodern}
\usepackage{iftex}
\ifPDFTeX
  \usepackage[T1]{fontenc}
  \usepackage[utf8]{inputenc}
  \usepackage{textcomp} % provide euro and other symbols
\else % if luatex or xetex
  \usepackage{unicode-math}
  \defaultfontfeatures{Scale=MatchLowercase}
  \defaultfontfeatures[\rmfamily]{Ligatures=TeX,Scale=1}
\fi

\titleformat{\section}
  {\normalfont\Large\scshape\bfseries}{\thesection}{1em}{}
  \titlespacing{\section}{0pt}{10pt}{0pt}
\titleformat{\subsection}
  {\normalfont\bfseries}{\thesection}{1em}{}
  \titlespacing{\subsection}{0pt}{6pt}{0pt}
\providecommand{\tightlist}{%
  \setlength{\itemsep}{0pt}\setlength{\parskip}{0pt}}\newenvironment{itemize*}%

  
\lhead{EC200 - Econometrics and Applications}

\setlength\parskip{0.10in}
\begin{document}
\thispagestyle{plain}
\singlespacing



EC200: Econometrics and Applications
%\vspace{1.5cm}
\begin{center}
\Large{\textbf{Annotated Bibliography}}\\
\end{center}
\bigskip


%{
%\setcounter{tocdepth}{3}
%\tableofcontents
%}

\hypertarget{objective}{%
\subsection*{Objective}\label{objective}}
\addcontentsline{toc}{subsection}{Objective}

The goal of this submission is to help you narrow and refine your
question while situating your work in the broader economics literature.
This will make writing your research paper much easier as well!

\hypertarget{whatis}{%
\subsection*{\texorpdfstring{What is an annotated
bibliography?\footnote{Pulled from
  \href{https://owl.purdue.edu/owl/general_writing/common_writing_assignments/annotated_bibliographies/index.html}{Purdue
  OWL}}}{What is an annotated bibliography?}}\label{whatis}}
\addcontentsline{toc}{subsection}{What is an annotated bibliography?}

\begin{quote}
A bibliography is a list of sources (books, journals, Web sites,
periodicals, etc.) one has used for researching a topic. Bibliographies
are sometimes called ``References'' or ``Works Cited'' depending on the
style format you are using. A bibliography usually just includes the
bibliographic information (i.e., the author, title, publisher, etc.).
\end{quote}

\begin{quote}
An annotation is a summary and/or evaluation. Therefore, an annotated
bibliography includes a summary and/or evaluation of each of the
sources.
\end{quote}

\hypertarget{mission}{%
\subsection*{What do I need to do}\label{mission}}
\addcontentsline{toc}{subsection}{What do I need to do}

Pick the idea you proposed that is most promising. You may refine it
based on feedback, further reflection, etc.

Based on that idea, identify and annotate \textbf{six} sources that are
relevant to your project. - At least \textbf{four} must be
peer-reviewed, academic journal articles. - At least \textbf{two} must
be from the list of economic journals included below.

For each one, include the following:

\begin{enumerate}
\def\labelenumi{\arabic{enumi}.}
\tightlist
\item
  Full bibliographic information, following MLA, APA, or Chicago style.
\item
  The annotations, written as a paragraph or as bullet points. These
  will include a few things:

  \begin{enumerate}
  \def\labelenumii{\alph{enumii}.}
  \tightlist
  \item
    Nature of source: peer-reviewed academic journal (what discipline),
    working paper, white paper (ie reports from major organizations),
    other
  \item
    Key findings or arguments of the source: It's in your interest to be
    quite detailed here (\emph{I like to use these to draw on when I
    write my paper})
  \item
    Assessment: How does it compare to other sources? (findings support
    or contrast)? Is the source biased or objective? What is the goal of
    the source?
  \item
    Reflection: Is this useful to your question? How does it help you
    shape your argument? How can you use this source in your project?
    (\emph{Here I will sometimes add sample sentences I will write})
  \end{enumerate}
\end{enumerate}

After this, write an expanded version of your idea proposal (just one
the idea you've chosen) that states your refined research question and
describes how your planned project fits into the literature you found.
This will be 2-3 paragraphs.

\textbf{Consult the \protect\hyperlink{rubric}{grading rubric} for
additional guidance!}

\hypertarget{acceptable-economics-journals}{%
\subsection{Acceptable economics
journals}\label{acceptable-economics-journals}}

At least two article must come from a relevant economics journal in the
\textbf{top 200} from the following
\href{https://ideas.repec.org/top/top.journals.simple.html}{RePeC list}.
If your topic is very specific, the \textbf{top 400} is also acceptable,
but you need to get prior approval from me.

\hypertarget{submission-requirements}{%
\subsection*{Submission requirements}\label{submission-requirements}}
\addcontentsline{toc}{subsection}{Submission requirements}

Submit the annotated bibliography plus summary as one document on
Blackboard.

If you are working in pairs, submit one bibliography for two people.

\hypertarget{tips}{%
\subsection*{Tips}\label{tips}}
\addcontentsline{toc}{subsection}{Tips}

\begin{itemize}
\item
  Search for your topic using
  \href{http://library.uvm.edu/research/research_databases}{EconLit}. If
  you at home, you can select \href{http://library.uvm.edu/}{``Connect
  Off Campus''} from the main library page
\item
  When determining if an article might be useful, start by focusing on
  the abstract only. Among those that pass your abstract test, then just
  read the introduction to see if they are still going to help.
\item
  When you've found an article or two that are useful, you can search
  forward and backward to find more!

  \begin{itemize}
  \tightlist
  \item
    Check the references section to find articles that were cited in
    your paper
  \item
    Use \href{https://scholar.google.com/}{Google Scholar} to find
    articles that cite your paper
  \end{itemize}
\item
  Note that working paper series are \emph{not} peer-reviewed journal
  articles.
  \href{https://www.nber.org/papers?page=1\&perPage=50\&sortBy=public_date}{NBER
  Working Papers}, for example, are an excellent resoucre but not
  peer-reviewed. Most reports from large organizations are not peer
  reviewed.
\end{itemize}

\hypertarget{rubric}{%
\subsection*{Rubric}\label{rubric}}
\addcontentsline{toc}{subsection}{Rubric}

You will receive up to 50 points on this assignment:

\begin{itemize}
\tightlist
\item
  Each source is worth 5 points (30 total), with one point per element
  listed above.
\item
  The idea summary is worth 10 points, with full credit granted if you
  present your research question, describe in words how you will answer
  it, and then describe how what you plan to do fits in with the
  literature you've reviewed.
\item
  The final 10 points are for meeting the source selection critera (4
  peer-reviewed, 2 top 200 economics journals)
\end{itemize}

\end{document}
